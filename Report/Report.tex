% #######################################
% ########### FILL THESE IN #############
% #######################################
\def\mytitle{Web Technology Coursework 1 Report}
\def\myauthor{Jonathan Binns}
\def\contact{40311703@live.napier.ac.uk}
\def\mymodule{Web Technologies (SET08101)}
% #######################################
% #### YOU DON'T NEED TO TOUCH BELOW ####
% #######################################
\documentclass[10pt, a4paper]{article}
\usepackage[a4paper,outer=1.5cm,inner=1.5cm,top=1.75cm,bottom=1.5cm]{geometry}
\onecolumn
\usepackage{graphicx}
\graphicspath{{./images/}}
%colour our links, remove weird boxes
\usepackage[colorlinks,linkcolor={black},citecolor={blue!80!black},urlcolor={blue!80!black}]{hyperref}
%Stop indentation on new paragraphs
\usepackage[parfill]{parskip}
%% Arial-like font
\IfFileExists{uarial.sty}
{
    \usepackage[english]{babel}
    \usepackage[T1]{fontenc}
    \usepackage{uarial}
    \renewcommand{\familydefault}{\sfdefault}
}{
    \GenericError{}{Couldn't find Arial font}{ you may need to install 'nonfree' fonts on your system}{}
    \usepackage{lmodern}
    \renewcommand*\familydefault{\sfdefault}
}
%Napier logo top right
\usepackage{watermark}
%Lorem Ipusm dolor please don't leave any in you final report ;)
\usepackage{lipsum}
\usepackage{xcolor}
\usepackage{listings}
%give us the Capital H that we all know and love
\usepackage{float}
%tone down the line spacing after section titles
\usepackage{titlesec}
%Cool maths printing
\usepackage{amsmath}
%PseudoCode
\usepackage{algorithm2e}

\titlespacing{\subsection}{0pt}{\parskip}{-3pt}
\titlespacing{\subsubsection}{0pt}{\parskip}{-\parskip}
\titlespacing{\paragraph}{0pt}{\parskip}{\parskip}
\newcommand{\figuremacro}[5]{
    \begin{figure}[#1]
        \centering
        \includegraphics[width=#5\columnwidth]{#2}
        \caption[#3]{\textbf{#3}#4}
        \label{fig:#2}
    \end{figure}
}

\lstset{
	escapeinside={/*@}{@*/}, language=C++,
	basicstyle=\fontsize{8.5}{12}\selectfont,
	numbers=left,numbersep=2pt,xleftmargin=2pt,frame=tb,
    columns=fullflexible,showstringspaces=false,tabsize=4,
    keepspaces=true,showtabs=false,showspaces=false,
    backgroundcolor=\color{white}, morekeywords={inline,public,
    class,private,protected,struct},captionpos=t,lineskip=-0.4em,
	aboveskip=10pt, extendedchars=true, breaklines=true,
	prebreak = \raisebox{0ex}[0ex][0ex]{\ensuremath{\hookleftarrow}},
	keywordstyle=\color[rgb]{0,0,1},
	commentstyle=\color[rgb]{0.133,0.545,0.133},
	stringstyle=\color[rgb]{0.627,0.126,0.941}
}

\thiswatermark{\centering \put(336.5,-38.0){\includegraphics[scale=0.8]{logo}} }
\title{\mytitle}
\author{\myauthor\hspace{1em}\\\contact\\Edinburgh Napier University\hspace{0.5em}-\hspace{0.5em}\mymodule}
\date{}
\hypersetup{pdfauthor=\myauthor,pdftitle=\mytitle}
\sloppy
% #######################################
% ########### START FROM HERE ###########
% #######################################
\begin{document}
    \maketitle
    \begin{abstract}
    Abstract goes here
    \end{abstract}

    
    \section{Introduction}
    \paragraph{}In this task the goal is to create a website that allows the user to encipher and decipher text in a choice of different ciphers. It should have a rewarding user experience that is easy to navigate and easy to both enter and retrieve the messages from. It should be achieved by having multiple files for each web page, a .HTML file which dictates the layout and text of the page, a .CSS file that dictates the styling of the page and finally a .js file that contains any JavaScript methods needed in the website.\\ \\ The two ciphers I decided to impliment are Bacon's cipher and the Rail Fence cipher. Bacon's cipher was invented by Sir Francis Bacon around 1576 to 1597 while he was in Paris\cite{Dawkins}. The main principle of this cipher is that each letter in the alphabet is given a 5 letter code made up of two values, for example 'b' becomes 'aaaab'. The main reason I chose this cipher is that it became the basis of important ciphers such as Morse Code and it was also the basis for how alphabetic characters can be represented in binary\cite{Dawkins}. The second cipher I chose was the rail fence cipher. It is an example of transposition or route cipher which was popular during the early history of cryptography \cite{Simmons}. The cipher works by splitting the message into a known number of parts, known as the key. The message is then split into these parts based on where each character is in the message, for example 'abcd' becomes 'acbd'. In order to decode the message the enciphered message would be written diagonally in a grid, revealing its message. I chose this cipher because it's doesn't follow a standard cipher structure, the letters are not replaced by another letter the entire string is rearranged. Testing my skills of working with strings and arrays of characters.   

	\section{Software Design}
	Software Design goes here
	
	
	\section{Implementation}
	Implementation goes here
	
	\section{Evaluation}
	Evaluation Goes Here
	
	\subsection{comparison Against the Requirements}
	Evaluation Against Requirements goes here
	
	\subsection{Improvements}
	Improvements go here
    
    
	\section{Personal Evaluation}
		
	\bibliographystyle{ieeetr}
	\bibliography{references}
		
\end{document}